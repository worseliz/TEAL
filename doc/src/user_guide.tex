\section{Introduction}
Tool for Economic AnaLysis is a RAVEN plug-in for performing cash flow analyses. 

\section{Installation}
These installation instructions assume that you have already successfully installed RAVEN. For RAVEN Installation instructions, visit https://github.com/idaholab/raven/wiki.
To install TEAL as a plug-in for RAVEN:
\begin{itemize}
    \item Navigate to raven/plugins
    \item Clone TEAL using git Clone
    \item Register TEAL plugin with raven: raven/scripts/install\_plugins.py -s TEAL
    \item To update TEAL, navigate to raven/plugins/TEAL and use git pull
\end{itemize}

\section{Command Line Basics}
Hopefully, through the installation process, you will have become familiar with some basics of navigating the command window. 

cd : Change Directory.
    Use this command to navigate to different folders. Ex. cd Documents/path/to/file
    You can navigate backwards in the directory by using cd .. Ex. cd ../../folder

ls : List Directory.
    Displays the names of files contained in the current directory. 

./ : Execute

conda activate raven\_libraries : This is a raven specific command which will allow you to use raven and its associated dependencies in the terminal.
It is good practice to activate these libraries before you start using TEAL. 
Execute a TEAL file:
To execute a TEAL program in conjunction with RAVEN, you must type the filepath to raven\_framework, followed by the RAVEN xml filename. The RAVEN file is the file that starts and ends with the <Simulation> block.
    ~/Documents/path/to/raven/raven\_framework CashFlow\_file\_name.xml 

\section{Getting Started}
Now that you know the basics, let's get started with the simulation of a single cash flow. We will begin by writing the TEAL (or <Economics>) block and then create the RAVEN (or <Simulation>) block.

Extensive Markup language (XML) is a language for encoding documents in a format that is both human-readable and machine readable. In the context of RAVEN, the XML format essentially groups and labels the data, variables and options required for the simulation.

Economics block:
This is the outer block of the TEAL file, which contains all other blocks required to run TEAL. Remember to add a second tag to close the block at the end. You can use an option here to specify the amount of output you want to see as the program runs.


\begin{lstlisting}[style=XML,morekeywords={class}]
    <Economics>

    </Economics>
\end{lstlisting}

Global block:
This block will contain information that applies to every component in the simulation. 
\begin{lstlisting}[style=XML,morekeywords={class}]
    <Economics>
        <Global>
        
        </Global>
    </Economics>
\end{lstlisting}

